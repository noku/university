%----------------------------------------------------------------------------------------
% PACKAGES AND DOCUMENT CONFIGURATIONS
%----------------------------------------------------------------------------------------

  \documentclass[12pt]{article}

  \usepackage{hyperref}
  \usepackage{fancyhdr} % Required for custom headers
  \usepackage{lastpage} % Required to determine the last page for the footer
  \usepackage{extramarks} % Required for headers and footers
  \usepackage[usenames,dvipsnames]{color} % Required for custom colors
  \usepackage{graphicx} % Required to insert images
  \usepackage{listings} % Required for insertion of code
  \usepackage{courier} % Required for the courier font
  \usepackage{lipsum} % Used for inserting dummy 'Lorem ipsum' text into the template
  \usepackage{wrapfig}
  \usepackage{color}
  \usepackage{lscape}

  \setlength\parindent{0pt} % Removes all indentation from paragraphs
  \renewcommand{\labelenumi}{\alph{enumi}.} % Make numbering in the itemize environment by letter rather than number (e.g. section 6)

  % Margins
  \topmargin=-0.7in
  \evensidemargin=0.2in
  \oddsidemargin=-0.2in
  \textwidth=7.0in
  \textheight=9.0in
  % \headsep=0.25in

  % \linespread{1.1} % Line spacing

  \definecolor{dkgreen}{rgb}{0,0.6,0}
  \definecolor{gray}{rgb}{0.5,0.5,0.5}
  \definecolor{mauve}{rgb}{0.58,0,0.82}
  \definecolor{greyish}{rgb}{0.96,0.96,0.96}

  \lstset{
    backgroundcolor=\color{greyish},   % choose the background color; you must add \usepackage{color} or \usepackage{xcolor}
    frame=tblr,
    numbers=left,                       % where to put the line-numbers; possible values are (none, left, right)
    numbersep=5pt,                   % how far the line-numbers are from the code
    numberstyle=\tiny\color{mygray}, % the style that is used for the line-numbers
    language=Prolog,
    aboveskip=3mm,
    belowskip=3mm,
    showstringspaces=false,
    columns=flexible,
    basicstyle={\footnotesize\ttfamily},
    numbers=none,
    numberstyle=\tiny\color{gray},
    keywordstyle=\color{blue},
    commentstyle=\color{dkgreen},
    stringstyle=\color{mauve},
    breaklines=true,
    breakatwhitespace=true
    tabsize=3
  }

  \begin{document}
  \begin{titlepage}

%----------------------------------------------------------------------------------------
% TITLE PAGE INFORMATION
%----------------------------------------------------------------------------------------
 \newcommand{\HRule}{\rule{\linewidth}{0.5mm}} % Defines a new command for the horizontal lines, change thickness here
  \begin{center} % Center everything on the page

  %----------------------------------------------------------------------------------------
  % HEADING SECTIONS
  %----------------------------------------------------------------------------------------
  \textsc{\large Faculty of Computers, Informatics and Microelectronics}\\[0.5cm]
  \textsc{\large Technical University of Moldova}\\[1.2cm] % Name of your university/college
  \vspace{35 mm}
  \textsc{\Large PLIA}\\[0.5cm] % Major heading such as course name
  %\textsc{\large Laboratory work \#1-3}\\[0.5cm] % Minor heading such as course title
  \textsc{\large Laboratory work \# 2}\\[0.5cm] % Minor heading such as course title

  %----------------------------------------------------------------------------------------
  % TITLE SECTION
  %----------------------------------------------------------------------------------------
  \vspace{10 mm}
  \HRule \\[0.4cm]
  { \large \bfseries Acquiring ideas about specific mechanisms of Prolog }\\[0.4cm] % Title of your document
  \HRule \\[1.5cm]

  %----------------------------------------------------------------------------------------
  % AUTHOR SECTION
  %----------------------------------------------------------------------------------------
      \vspace{30mm}

      \begin{minipage}{0.4\textwidth}
      \begin{flushleft} \large
      \emph{Author:}\\
      Petru \textsc{Negrei} % Your name
      \end{flushleft}
      \end{minipage}
      ~
      \begin{minipage}{0.4\textwidth}
      \begin{flushright} \large
      \emph{Supervisor:} \\
      L. \textsc{Luchianov} % Supervisor's Name
      \end{flushright}
      \end{minipage}\\[4cm]

      \vspace{5 mm}
      % If you don't want a supervisor, uncomment the two lines below and remove the section above
      %\Large \emph{Author:}\\
      %John \textsc{Smith}\\[3cm] % Your name

      %----------------------------------------------------------------------------------------
      % DATE SECTION
      %----------------------------------------------------------------------------------------

      {\large December 2014}\\[3cm] % Date, change the \today to a set date if you want to be precise

      %----------------------------------------------------------------------------------------
      % LOGO SECTION
      %----------------------------------------------------------------------------------------

      %\includegraphics{Logo}\\[1cm] % Include a department/university logo - this will require the graphicx package

      %----------------------------------------------------------------------------------------

      \vfill % Fill the rest of the page with whitespace
      \end{center}
      \end{titlepage}

      % \newpage
      % \tableofcontents
      % \newpage

%----------------------------------------------------------------------------------------
% Introduction
%----------------------------------------------------------------------------------------

  \section{Introduction}

  \subsection{Topic}

   Acquiring ideas about specific mechanisms of Prolog

  \subsection{Tasks}

  1. Launch the execution work program established in laboratory work 1 and investigated changes:

  \begin{itemize}
    \item By changing the order of sentences facts;
    \item By changing the order of sentences rules; (two versions)
    \item By changing the order of sentences facts; (two versions)
    \item Make conclusions.
  \end{itemize}

  \vspace{0.5cm}

  2. Solve the following problems proposed and will follow their proper implementation.
  
  \begin{itemize}
    \item Develop and test a program to determine a minimum value of two numbers (X and Y), without using predicate cut.
    \item Develop and test a program to determine a minimum value of two numbers (X and Y), using red and green cut predicates.
    \item What will be the program answers to given task, do a comparative analysis between predicates using the knowledge base cut in space and space purposes.
    \item Two children play a game in a tennis tournament if they have the same age. Whether subsequent children and their ages: \\
        \textbf{child (peter, 9). child (paul, 10). child (chris, 9). child (sesame, 9). } \\ 
     Define a predicate which shows all pairs of children who can play a game in a tennis tournament.
  \end{itemize}

 3. Enter the appropriate changes in the program (2.1), using green cut at least two rules in the knowledge base. \\ 

 4. Enter the appropriate changes in the program (2.1), using red cut rules in the knowledge base. Write conclusions.

  \subsubsection{Report}

  Report will containt a short description of work done, and will present necessary information
  about tools, algorithms used or studied.

%----------------------------------------------------------------------------------------
% Implementation
%----------------------------------------------------------------------------------------

  \section{Implementation}

  \subsection{}
      
  \subsubsection{}
  
  After changing the order of rules I observed that no changes occur in the program, but 
  when I changed the order of facts, I noticed that it changes the order of the showed results.

  Thus when changing the rules we change the order in which the results will appear to the user.

  \subsection{}

  \subsubsection{}

    Develop and test a program to determine a minimum value of two numbers (X and Y), without using predicate cut.
  
    \begin{lstlisting}
    min(X,Y,X):- X =< Y.
    min(X,Y,Y):- X > Y.
    \end{lstlisting}

    \subsubsection{}

    Develop and test a program to determine a minimum value of two numbers (X and Y), using cut cut predicate red and green.

    \begin{lstlisting}
    % green cut
    min(X, Y, X) :- X =< Y, !.
    min(X, Y, Y) :- X > Y.
    \end{lstlisting}

    \begin{lstlisting}
    % red cut
    min(X, Y, X) :- X =< Y, !. 
    min(X, Y, Y).
    \end{lstlisting}

    \subsubsection{}

    What will be the program answers to given task, do a comparative analysis between predicates using the knowledge base cut in space and space purposes \\
   
      In the first case of \textbf{p(X)}. we have the following answers: \textbf{X = 1; X = 2}. We have the answers because it enters in backtraking and check all the time 
      until it reaches the clause p(2): - !. and stops because the cut predicate was used, which ignores the responses that follow. \\

      For the purposes of \textbf{p(X), p(Y).} the answers will be: \\

      X = Y, Y = 1; \\
      X = 1, Y = 2; \\
      X = 2, Y = 1; \\
      X = Y, Y = 2. \\

      For the purposes of \textbf{p(X),!,p(Y).} because of cut the answers will be: \\

      X = Y, Y = 1; \\
      X = 1, Y = 2, 

      \subsubsection{}

     Two children play a game in a tennis tournament if they have the same age. Whether subsequent children and their ages: \\

    \begin{lstlisting}
    child(peter, 9). 
    child(paul, 10). 
    child(chris, 9). 
    child(sesame, 9).
    \end{lstlisting}

     Define a predicate which shows all pairs of children who can play a game in a tennis tournament.

    \begin{lstlisting}
    perechi(A,B,C):-copil(A,C),copil(B,C) ,A @<B, !.
    \end{lstlisting}


    \subsection{List}
        
    Write a predicate called \textit{descomp(N, Lista)} witch will receive an integer number N and 
    return a list of prime facotors of this number.

    \begin{lstlisting}
    % Determine the prime factors of a given positive integer. 

    % descomp(N, L) :-  L is the list of prime factors of N.
    %    (integer,list) (+,?)

    descomp(N,L) :- N > 0,  descomp(N,L,2).

    % descomp(N,L,K) :- L is the list of prime factors of N. It is 
    % known that N does not have any prime factors less than K.
    
    descomp(1,[],_) :- !.
    descomp(N,[F|L],F) :-   % N is multiple of F
       R is N // F, N =:= R * F, !, descomp(R,L,F).
    descomp(N,L,F) :- 
       next_factor(N,F,NF), descomp(N,L,NF).        % N is not multiple of F
       
    % next_factor(N,F,NF) :- when calculating the prime factors of N
    %    and if F does not divide N then NF is the next larger candidate to
    %    be a factor of N.

    next_factor(_,2,3) :- !.
    next_factor(N,F,NF) :- F * F < N, !, NF is F + 2.
    next_factor(N,_,N).                                 % F > sqrt(N)
    \end{lstlisting}


    \section{Conclusion}
    \label{sec:conclusion}

    As a result of this laboratory work I analyzed theoretical material and became acquainted with some features of Prolog. 
    In carrying out the work there have been solved some problems with using predicate cut red and green.
    After solving can say that the use of the green cut increase the efficiency of the program, also the use of 
    the clauses use the order does not matter, the result will not depend on it. 
    When using red cut it changes procedural significance of the program, in case the order is changed clauses will have an incorrect result.


\end{document}