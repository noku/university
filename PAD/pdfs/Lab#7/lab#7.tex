%----------------------------------------------------------------------------------------
% PACKAGES AND DOCUMENT CONFIGURATIONS
%----------------------------------------------------------------------------------------

  \documentclass[12pt]{article}

  \usepackage{hyperref}
  \usepackage{fancyhdr} % Required for custom headers
  \usepackage{lastpage} % Required to determine the last page for the footer
  \usepackage{extramarks} % Required for headers and footers
  \usepackage[usenames,dvipsnames]{color} % Required for custom colors
  \usepackage{graphicx} % Required to insert images
  \usepackage{listings} % Required for insertion of code
  \usepackage{courier} % Required for the courier font
  \usepackage{lipsum} % Used for inserting dummy 'Lorem ipsum' text into the template
  \usepackage{wrapfig}
  \usepackage{color}
  \usepackage{lscape}

  \setlength\parindent{0pt} % Removes all indentation from paragraphs
  \renewcommand{\labelenumi}{\alph{enumi}.} % Make numbering in the itemize environment by letter rather than number (e.g. section 6)
  \lstset{basicstyle=\ttfamily\footnotesize,breaklines=true}

  % Margins
  \topmargin=-0.4in
  \evensidemargin=0.2in
  \oddsidemargin=-0.2in
  \textwidth=7.0in
  \textheight=9.0in
  % \headsep=0.25in

  % \linespread{1.1} % Line spacing

  \definecolor{dkgreen}{rgb}{0,0.6,0}
  \definecolor{gray}{rgb}{0.5,0.5,0.5}
  \definecolor{mauve}{rgb}{0.58,0,0.82}
  \definecolor{greyish}{rgb}{0.96,0.96,0.96}

  \lstset{
    backgroundcolor=\color{greyish},   % choose the background color; you must add \usepackage{color} or \usepackage{xcolor}
    frame=tblr,
    numbers=left,                       % where to put the line-numbers; possible values are (none, left, right)
    numbersep=5pt,                   % how far the line-numbers are from the code
    numberstyle=\tiny\color{mygray}, % the style that is used for the line-numbers
    language=Ruby,
    aboveskip=3mm,
    belowskip=3mm,
    showstringspaces=false,
    columns=flexible,
    basicstyle={\footnotesize\ttfamily},
    numbers=none,
    numberstyle=\tiny\color{gray},
    keywordstyle=\color{blue},
    commentstyle=\color{dkgreen},
    stringstyle=\color{mauve},
    breaklines=true,
    breakatwhitespace=true
    tabsize=1
  }

  \begin{document}
  \begin{titlepage}

%----------------------------------------------------------------------------------------
% TITLE PAGE INFORMATION
%----------------------------------------------------------------------------------------
 \newcommand{\HRule}{\rule{\linewidth}{0.5mm}} % Defines a new command for the horizontal lines, change thickness here
  \begin{center} % Center everything on the page

  %----------------------------------------------------------------------------------------
  % HEADING SECTIONS
  %----------------------------------------------------------------------------------------
  \textsc{\large Faculty of Computers, Informatics and Microelectronics}\\[0.5cm]
  \textsc{\large Technical University of Moldova}\\[1.2cm] % Name of your university/college
  \vspace{35 mm}
  \textsc{\Large PAD}\\[0.5cm] % Major heading such as course name
  %\textsc{\large Laboratory work \#1-3}\\[0.5cm] % Minor heading such as course title
  \textsc{\large Laboratory work \# 7}\\[0.5cm] % Minor heading such as course title

  %----------------------------------------------------------------------------------------
  % TITLE SECTION
  %----------------------------------------------------------------------------------------
  \vspace{10 mm}
  \HRule \\[0.4cm]
  { \large \bfseries  REST.}\\[0.4cm] % Title of your document
  \HRule \\[1.5cm]

  %----------------------------------------------------------------------------------------
  % AUTHOR SECTION
  %----------------------------------------------------------------------------------------
      \vspace{25mm}

      \begin{minipage}{0.4\textwidth}
      \begin{flushleft} \large
      \emph{Authors:}\\
      \textbf{Petru \textsc{Negrei}} \\
      Victor \textsc{Vasilica}
      \end{flushleft}
      \end{minipage}
      ~
      \begin{minipage}{0.4\textwidth}
      \begin{flushright} \large
      \emph{Supervisor:} \\
      D. \textsc{Ciorba} % Supervisor's Name
      \end{flushright}
      \end{minipage}\\[4cm]

      \vspace{5 mm}
      % If you don't want a supervisor, uncomment the two lines below and remove the section above
      %\Large \emph{Author:}\\
      %John \textsc{Smith}\\[3cm] % Your name

      %----------------------------------------------------------------------------------------
      % DATE SECTION
      %----------------------------------------------------------------------------------------

      {\large December 2014}\\[3cm] % Date, change the \today to a set date if you want to be precise

      %----------------------------------------------------------------------------------------
      % LOGO SECTION
      %----------------------------------------------------------------------------------------

      %\includegraphics{Logo}\\[1cm] % Include a department/university logo - this will require the graphicx package

      %----------------------------------------------------------------------------------------

      \vfill % Fill the rest of the page with whitespace
      \end{center}
      \end{titlepage}

      % \newpage
      % \tableofcontents
      % \newpage

%----------------------------------------------------------------------------------------
% Introduction
%----------------------------------------------------------------------------------------

  \section{Introduction}

  \subsection{Topic}

  Describe the REST service.

%----------------------------------------------------------------------------------------
% Implementation
%----------------------------------------------------------------------------------------
  
  \section{Description}


  \subsection{Definition}
      
    Representational state transfer (REST) is an abstraction of the architecture of the World Wide Web; more precisely, 
    REST is an architectural style consisting of a coordinated set of architectural constraints applied to components, 
    connectors, and data elements, within a distributed hypermedia system.

    REST ignores the details of component implementation and protocol syntax in order to focus on the roles of components, 
    the constraints upon their interaction with other components, and their interpretation of significant data elements.

    The REST architectural style is also applied to the development of web services. One can characterize web services as "RESTful" 
    if they conform to the constraints described in the architectural constraints section.

    \subsection{Content}
        
    During previous laboratory works we implemented a distributed system based on Restful principles and design 
    the requests and responses based on that.

    \subsubsection{Structure}

    A uniform interface separates clients from servers. This separation of concerns means that, for example, 
    clients are not concerned with data storage, which remains internal to each server, so that the portability of 
    client code is improved. Servers are not concerned with the user interface or user state, so that servers can be 
    simpler and more scalable. Servers and clients may also be replaced and developed independently, 
    as long as the interface between them is not altered.

    \subsubsection{Stateless}

    The client–server communication is further constrained by no client context being stored on the server between requests. 
    Each request from any client contains all the information necessary to service the request, and session state is held in the client. 
    The session state can be transferred by the server to another service such as a database to maintain a persistent state
    for a period and allow authentication. The client begins sending requests when it is ready to make the transition to a new
    state. While one or more requests are outstanding, the client is considered to be in transition. The representation of each
     application state contains links that may be used the next time the client chooses to initiate a new state-transition
        
    \subsubsection{Cacheable}
        
    As on the World Wide Web, clients can cache responses. Responses must therefore, implicitly or explicitly, 
    define themselves as cacheable, or not, to prevent clients from reusing stale or inappropriate data in response to 
    further requests. Well-managed caching partially or completely eliminates some client–server interactions, further 
    improving scalability and performance.

    \subsubsection{Uniform interface}

    The uniform interface constraint is fundamental to the design of any REST service.
    The uniform interface simplifies and decouples the architecture, which enables each part to evolve independently. 
      

    \subsection{HTTP requests}

    HTTP based RESTful APIs are defined with these aspects.
    The following list shows the HTTP methods that are typically used to implement a RESTful API.

    \textbf{http://example.com/resources}.

    \begin{itemize}
      \item \textbf{GET} - List the URIs and perhaps other details of the collection's members.
      \item \textbf{PUT} - Replace the entire collection with another collection.
      \item \textbf{POST} - Create a new entry in the collection. The new entry's URI is assigned automatically and is usually returned by the operation.
      \item \textbf{DELETE} - Delete the entire collection.
    \end{itemize}

    \textbf{http://example.com/resources/item17}.

    \begin{itemize}
      \item \textbf{GET} - Retrieve a representation of the addressed member of the collection, expressed in an appropriate Internet media type.
      \item \textbf{PUT} - Replace the addressed member of the collection, or if it doesn't exist, create it.
      \item \textbf{POST} - Not generally used. Treat the addressed member as a collection in its own right and create a new entry in it.
      \item \textbf{DELETE} - Delete the addressed member of the collection.
    \end{itemize}

   \section{References}

   \begin{itemize}
      \item Wikipedia \url{http://www.wikiwand.com/en/Representational_state_transfer}
   \end{itemize}

\end{document}